%%%%%%%%%%%%%%%%%%%% author.tex %%%%%%%%%%%%%%%%%%%%%%%%%%%%%%%%%%%
%
% sample root file for your "contribution" to a contributed volume
%
% Use this file as a template for your own input.
%
%%%%%%%%%%%%%%%% Springer %%%%%%%%%%%%%%%%%%%%%%%%%%%%%%%%%%


% RECOMMENDED %%%%%%%%%%%%%%%%%%%%%%%%%%%%%%%%%%%%%%%%%%%%%%%%%%%
\documentclass[graybox]{svmult}

% choose options for [] as required from the list
% in the Reference Guide

\usepackage{type1cm}        % activate if the above 3 fonts are
                            % not available on your system
%
\usepackage{makeidx}         % allows index generation
\usepackage{graphicx}        % standard LaTeX graphics tool
                             % when including figure files
\usepackage{multicol}        % used for the two-column index
\usepackage[bottom]{footmisc}% places footnotes at page bottom


\usepackage{newtxtext}       % 
\usepackage[varvw]{newtxmath}       % selects Times Roman as basic font

% see the list of further useful packages
% in the Reference Guide

\makeindex             % used for the subject index
                       % please use the style svind.ist with
                       % your makeindex program

%%%%%%%%%%%%%%%%%%%%%%%%%%%%%%%%%%%%%%%%%%%%%%%%%%%%%%%%%%%%%%%%%%%%%%%%%%%%%%%%%%%%%%%%%

\begin{document}

\title*{City growth regularities}

\author{Name of First Author and Name of Second Author}

\institute{Name of First Author \at Name, Address of Institute, \email{name@email.address}
\and Name of Second Author \at Name, Address of Institute \email{name@email.address}}


\maketitle



\abstract{}

\section{Introduction}






%
%\begin{acknowledgement}
%\end{acknowledgement}
%

%\section*{Appendix}
%\addcontentsline{toc}{section}{Appendix}



\bibliographystyle{spbasic}
\bibliography{biblio}


\end{document}





%%%%% Template


%
\begin{equation}
a \times b = c\;,
\end{equation}
%

\begin{eqnarray}
\left|\nabla U_{\alpha}^{\mu}(y)\right| &\le&\frac1{d-\alpha}\int
\left|\nabla\frac1{|\xi-y|^{d-\alpha}}\right|\,d\mu(\xi) =
\int \frac1{|\xi-y|^{d-\alpha+1}} \,d\mu(\xi)  \\
&=&(d-\alpha+1) \int\limits_{d(y)}^\infty
\frac{\mu(B(y,r))}{r^{d-\alpha+2}}\,dr \le (d-\alpha+1)
\int\limits_{d(y)}^\infty \frac{r^{d-\alpha}}{r^{d-\alpha+2}}\,dr
\label{eq:01}
\end{eqnarray}


% For tables use
%
\begin{table}[!t]
\caption{Please write your table caption here}
\label{tab:1}       % Give a unique label
%
% Follow this input for your own table layout
%
\begin{tabular}{p{2cm}p{2.4cm}p{2cm}p{4.9cm}}
\hline\noalign{\smallskip}
Classes & Subclass & Length & Action Mechanism  \\
\noalign{\smallskip}\svhline\noalign{\smallskip}
Translation & mRNA$^a$  & 22 (19--25) & Translation repression, mRNA cleavage\\
Translation & mRNA cleavage & 21 & mRNA cleavage\\
Translation & mRNA  & 21--22 & mRNA cleavage\\
Translation & mRNA  & 24--26 & Histone and DNA Modification\\
\noalign{\smallskip}\hline\noalign{\smallskip}
\end{tabular}
$^a$ Table foot note (with superscript)
\end{table}
%



\begin{figure}[t]
\sidecaption[t]
% Use the relevant command for your figure-insertion program
% to insert the figure file.
% For example, with the option graphics use
\includegraphics[scale=.65]{figure}
%
% If no graphics program available, insert a blank space i.e. use
%\picplace{5cm}{2cm} % Give the correct figure height and width in cm
%
%\caption{Please write your figure caption here}
\caption{If the width of the figure is less than 7.8 cm use the \texttt{sidecapion} command to flush the caption on the left side of the page. If the figure is positioned at the top of the page, align the sidecaption with the top of the figure -- to achieve this you simply need to use the optional argument \texttt{[t]} with the \texttt{sidecaption} command}
\label{fig:2}       % Give a unique label
\end{figure}



